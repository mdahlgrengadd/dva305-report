
% DVA305 Rapportmall (LaTeX) — pixel/typography-matched to the provided Word template as closely as LaTeX allows.
% Compile with: lualatex (recommended) or xelatex.
\documentclass[11pt,a4paper]{article}

% ---------- Language ----------
\usepackage[swedish]{babel}
\usepackage{csquotes}

% ---------- Page geometry (A4; Word template margins) ----------
\usepackage[a4paper,
  top=3.0cm,
  bottom=3.5cm,
  left=3.0cm,
  right=3.0cm
]{geometry}

% ---------- Fonts (Word template uses Georgia body + Arial headings) ----------
\usepackage{fontspec}
\setmainfont{Georgia}
\setsansfont{Arial}
\usepackage{microtype} % improves glyph spacing/kerning (LaTeX side)

% ---------- Paragraph metrics (Word default: 1.15 line spacing, ~10pt after, no indent) ----------
\usepackage{setspace}
\setstretch{1.15}
\setlength{\parindent}{0pt}
\setlength{\parskip}{10pt}

% ---------- Headings (sans/Arial; compact like Word) ----------
\usepackage{titlesec}
\titleformat{\section}
  {\sffamily\bfseries\fontsize{12pt}{14pt}\selectfont}
  {\thesection}{0.75em}{}
\titleformat{\subsection}
  {\sffamily\bfseries\fontsize{12pt}{14pt}\selectfont}
  {\thesubsection}{0.75em}{}
\titlespacing*{\section}{0pt}{12pt}{6pt}
\titlespacing*{\subsection}{0pt}{12pt}{6pt}

% ---------- TOC + List of Figures (Word-like leaders and spacing) ----------
\usepackage{tocloft}
\renewcommand{\contentsname}{Innehållsförteckning}
\renewcommand{\listfigurename}{Figurförteckning}
\setlength{\cftbeforesecskip}{2pt}
\renewcommand{\cftsecleader}{\cftdotfill{\cftdotsep}}
\renewcommand{\cftsecfont}{\normalfont}
\renewcommand{\cftsecpagefont}{\normalfont}
\renewcommand{\cfttoctitlefont}{\sffamily\bfseries\fontsize{12pt}{14pt}\selectfont}
\renewcommand{\cftloftitlefont}{\sffamily\bfseries\fontsize{12pt}{14pt}\selectfont}

% ---------- Hyperlinks (optional; keeps PDF usable without changing look much) ----------
\usepackage[hidelinks]{hyperref}

% ---------- Figures ----------
\usepackage{graphicx}
\usepackage{caption}
\captionsetup{font=small,labelfont=bf}

% ---------- References (IEEE numeric brackets) ----------
\usepackage[backend=biber,style=ieee]{biblatex}
\addbibresource{references.bib}

% ---------- Title page helpers ----------
\newcommand{\CourseCode}{DVA305}
\newcommand{\CourseName}{Information – kunskap – vetenskap – etik}
\newcommand{\Semester}{[VT/HT] 202X}

\begin{document}

% =======================
% Title page (not in TOC)
% =======================
\thispagestyle{empty}

% Word's first line "Rapportmall" is styled as Title (Arial 48pt bold)
{\sffamily\bfseries\fontsize{48pt}{54pt}\selectfont Rapportmall\par}

\vspace{18pt}

% The template then shows course metadata in normal body font.
{\CourseCode\par}
{\CourseName\par}
{\Semester\par}

\vfill

% --- Replace the placeholders below ---
\begin{center}
{\sffamily\bfseries\fontsize{18pt}{22pt}\selectfont Titel\par}
\vspace{8pt}
Författare 1, Författare 2\par
Kurs: \CourseCode\ (\CourseName)\par
E-post: namn1@mdh.se, namn2@mdh.se\par
Ort och datum: Västerås, \today\par
\vspace{12pt}
% \includegraphics[width=0.6\textwidth]{title-image} % valfri bild på titelbladet
\end{center}

\newpage

% =======================
% Abstract (not in TOC)
% =======================
\thispagestyle{empty}
{\sffamily\bfseries\fontsize{12pt}{14pt}\selectfont Sammanfattning\par}
% Riktmärke: 200–250 ord. Ska kunna läsas fristående.
Skriv en kort, kompakt sammanfattning av hela rapporten (ca 200–250 ord). Den ska
introducera området, presentera frågeställningen översiktligt, beskriva hur ni angripit den
och sammanfatta de viktigaste resultaten och slutsatserna. Undvik detaljer och undvik att beskriva
rapportens disposition.

\newpage

% ==========================================
% TOC + List of Figures (not themselves in TOC)
% ==========================================
\thispagestyle{empty}
\tableofcontents
\newpage

\thispagestyle{empty}
\listoffigures
\newpage

% =========================================================
% Main matter: start page numbering to match Word example
% Word sample shows "Inledning" starting at page 5.
% =========================================================
\pagenumbering{arabic}
\setcounter{page}{5}

\section{Inledning}
I inledningen ska ni introducera ämnet för ert arbete och motivera varför det är viktigt.
Inledningen ska vara allmän och syftar till att ge läsaren en introduktion till ämnet och leder
läsaren in i texten.

Följande bör ingå:
\begin{itemize}
  \item Presentation av området och ämnet för arbetet (tidigt, fånga intresset).
  \item Presentation av ert arbete inklusive syfte och frågeställning (på hög nivå).
  \item Motivation till varför uppgiften är intressant och varför frågeställningen är relevant.
\end{itemize}

Ingen specialiserad terminologi eller matematik bör ingå i inledningen.

\section{Bakgrund}
Redogör för sådan kunskap som läsaren behöver för att förstå ert arbete och ert bidrag.
Placera in ert arbete i ett vetenskapligt sammanhang och jämför övergripande med tidigare arbeten.

\section{Frågeställning}
Formulera och precisera frågeställningen. Beskriv antaganden och begränsningar.

\section{Metod}
Beskriv vilken vetenskaplig metod ni använt och hur ni gått tillväga för att besvara frågeställningen.
Beskriv både \emph{vad} och \emph{varför}.

\section{Forskningsetiskt ställningstagande}
Beskriv eventuella forskningsetiska aspekter. Om inga uppstår, skriv tydligt att ni inte identifierat
etiska frågor för er frågeställning/metod.

\section{Utförande}
Byt gärna rubriken till en mer passande rubrik för ert arbete. Strukturen ska vara tydlig och
experiment ska gå att upprepa (annars lägg detaljer i bilaga).

\section{Resultat}
Presentera och analysera resultaten så att de kan bedömas av en läsare.

\section{Diskussion}
Tolka resultaten, diskutera signifikans, konsekvenser, begränsningar och koppla till tidigare arbete.

\section{Relaterat arbete}
Beskriv relevant existerande forskning mer på djupet och jämför angreppssätt/metodik/resultat
med ert eget.

\section{Slutsatser}
Summera rapporten och presentera slutsatser (utan nya detaljer). En expert ska kunna läsa detta avsnitt fristående.

\section{Framtida arbete}
Diskutera förbättringar, utvidgningar och möjliga nya frågeställningar.

\section{Referenser}
Exempel på IEEE-referens i text: \cite{exempel}.

\printbibliography

\appendix
\section{Bilagor}
Övrig information som inte passar i huvudtexten (t.ex. källkod, testdata, extra figurer).

\end{document}
