% ==========================================
% 3. FRÅGESTÄLLNING [cite: 61, 190]
% ==========================================
%\section{Hypotes}
%Vi förväntar oss att SDXL latenta representation ska gå att komprimera ytterliggare med vanliga metoder såsom AVIF eftersom dess kanaler inte bara tycks bestå av okorrellerat “brus”. Detta verifieras genom att latenten omformas till ett bildlikt format (RGBA) och öppnas i ett verktyg som GIMP där det då går att urskilja tydliga bildstrukturer. Det tyder på att latenten innehåller spatialt sammanhängande information (kanter, större former, färgövergångar) och därmed uppvisar redundans som nämnda kompressionsalgoritmer är konstruerade att utnyttja. 


\section*{Frågeställning}
\addcontentsline{toc}{section}{Frågeställning} 
%I det här avsnittet formulerar och preciserar ni er frågeställning. Frågeställningen är den vetenskapliga rapportens viktigaste och mest centrala del.

%Beskriv frågeställningen på ett tydligt sätt, både på hög nivå och i detalj. Redogör för antaganden och begränsningar, samtidigt är en välformulerad frågeställning i sig avgränsande.
Uppvisar SDXL:s latenta representation tillräckligt med 'bildlika' egenskaper för att vanliga bildkomprimeringsmetoder (AVIF/PNG) ska fungera för att minska lagringsstorleken utan att bildkvaliteten faller samman vid rekonstruktion? 
