% ==========================================
% 6. UTFÖRANDE [cite: 75, 204]
% ==========================================
\section*{Utförande}
\addcontentsline{toc}{section}{Utförande} 
\textit{(Obs: Ni ska inte använda rubriken ovan, utan ersätta den med lämplig rubrik beroende på ert arbete.)}
    \vspace{0.5cm} % tweak this

%Efter avsnitten ovan följer nu en beskrivning av vad ni gjort. Strukturen ska göras tydlig genom avsnittsrubrikerna. Det är viktigt med en klar och tydlig logisk struktur och ett berättande flöde.

%Ni ska ha med avancerad bakgrundskunskap som är nödvändig för att förstå hur ni löst uppgiften, och definiera hypoteser och viktiga begrepp. Beskrivning av experiment ska vara sådan att det ska gå att upprepa experimenten. Om en sådan beskrivning blir väldigt lång och detaljerad kan ni lägga den i en bilaga.
