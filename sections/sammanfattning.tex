% ==========================================
% SAMMANFATTNING [cite: 14, 153]
% ==========================================
\thispagestyle{plain} % Simple page number
\section*{Sammanfattning}
Det här avsnittet ska helt enkelt vara en sammanfattning av hela rapporten. En lämplig omfattning är c:a 200 – 250 ord. En bra tumregel är att sammanfattningen ska hållas så kort det går, den ska vara kompakt men tydlig, informativ och väcka intresse.

Summera allt det som är väsentligt i rapporten och presentera de viktigaste resultaten och slutsatserna. Följande bör ingå:
\begin{itemize}
    \item Presentation och introduktion av området för arbetet.
    \item Översiktlig presentation av frågeställningen.
    \item Generell beskrivning av arbetet, hur ni angripit frågeställningen och vad ni har gjort.
    \item Sammanfattning av resultat och slutsatser.
\end{itemize}

Det ska inte finnas någon beskrivning av hur rapporten är uppställd och inte några detaljer i sammanfattningen. Sammanfattningen ska kunna läsas helt fristående från resten av rapporten och vända sig till en ganska bred målgrupp läsare. Den ska ge en bra grund för att en läsare ska kunna bedöma om hen är intresserad av att läsa hela rapporten. 

När hela rapporten är klar bör ni granska och vid behov revidera sammanfattningen så att den överensstämmer med rapporten.
